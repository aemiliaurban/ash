\section*{Abstract}

Flow cytometry allows inexpensive monitoring of large and diverse cell populations using fluorescent markers, providing immense applications in studying biological properties of blood and tissues as well as diagnostics in the clinical setting. Recent methodological advances highlight automatic clustering as a tool of choice for data analysis, and many clustering algorithms were developed for various use cases. However, the applicability of such algorithms in biology and medicine remains challenging unless the tools expose user-friendly, interactive interfaces that are accessible to domain experts. The goal of the thesis is to review the available methods that allow such interaction and supervision of the clustering process by the user, specifically focusing on interfaces desirable in clinical settings that do not require the user to interact with scripting or programming environments. As the main practical result, the thesis should design a new tool that builds upon previously developed methodology (iDendro, gMHCA), allowing the application of the researched methodology on realistic datasets. By using proper data visualization techniques, the end user should be able to interact with the dataset in a way that is both intuitive and useful for producing biologically relevant results. The thesis should also review data exchange formats that would be suitable for working with various other kinds of clustering algorithms. 

\bigskip

\begin{tabular}{lp{8.6cm}}
		\textbf{Keywords} & \Keywords \\
 		& \\
 		\textbf{Author's e-mail} & \texttt{\href{mailto:\Email}{\Email}}\\
		\textbf{Supervisor's e-mail} & \texttt{\href{mailto:\EmailSup}{\EmailSup}}\\
\end{tabular}

\bigskip

\section*{Abstrakt}\label{abstract}

Flow cytometrie umožňuje levné monitorování velkých a různorodých buněčných populací za použití fluorescentních markerů, díky čemuž je využívána jak ve výzkumu biologických vlastností krve a tkání, tak i jako diagnostický názor v klinickém prostředí. Nedávné posuny v metodologii zvýrazňují automatické clusterování jako nejvhodnější nástroj pro analýzu dat, a mnohé clusterovací algoritmy byly vyvinuty pro různé případy užití. Pokud však dostupné nástroje nebudou interaktivní a snadno přístupné uživateli, praktické využití v medicíně a biologii nebude dostávat svého potenciálu. Cílem práce je zhodnotit dostupné metody, které umožňují interakci a supervizi clusterovacího procesu uživatelem se zaměřením na žádoucí rozhraní v klinickém prostředí, které nevyžaduje, aby uživatel interagoval se skriptujícími či programovacími rozhraními. Práce si klade za hlavní pracovní cíl nadesignování nového nástroje, který nastavuje na dříve vyvinutou metodologii (iDendro, gMHCA) a umožňuje aplikaci zkoumaných nástrojů na realistickém datasetu. Za použití technik z oblasti vizualizace dat by měl být uživatel schopný interagovat s datasetem který je zároveň intuitivní a užitečný pro produkování biologicky relevantních výsledků. Práce by také měla shrnout formáty výmeň dat, které by byly vhodné pro práci s jinými typy clusterovacích algoritmů. 


\bigskip

\begin{tabular}{lp{7.7cm}}
		\textbf{Kl\'i\v cov\'a slova} & \Klic \\
 		& \\
 		\textbf{E-mail autora} & \texttt{\href{mailto:\Email}{\Email}}\\
		\textbf{E-mail vedouc\'iho pr\'ace} & \texttt{\href{mailto:\EmailSup}{\EmailSup}}\\
\end{tabular}

